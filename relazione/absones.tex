\documentclass[a4paper,11pt]{article}

\usepackage[italian]{babel}
\usepackage[utf8]{inputenc}
\usepackage{amsmath}
\usepackage{graphicx}
\usepackage{amssymb}
\usepackage{amstext}
\usepackage{xcolor}
\usepackage[colorinlistoftodos]{todonotes}
\usepackage[displaymath, mathlines]{lineno}
\usepackage{authblk}
\usepackage{multicol}
\usepackage{tikz}

% FANCY TITLE AND ABSTRACT IS LIT
\usepackage{lipsum}
\usepackage{authblk}
\usepackage[top=2cm, bottom=2cm, left=2cm, right=2cm]{geometry}
\usepackage{fancyhdr}
%
\pagestyle{fancy}
%
\renewenvironment{abstract}
{\begin{quote}
\noindent \rule{\linewidth}{.5pt}\par{\bfseries \abstractname.}}
{\medskip\noindent \rule{\linewidth}{.5pt}
\end{quote}
}
%
\makeatletter
\renewcommand\@maketitle{%
\hfill
\begin{minipage}{0.95\textwidth}
\vskip 2em
\let\footnote\thanks 
{\LARGE \@title \par }
\vskip 1.5em
{
  \begin{minipage}{.5\textwidth}
    \textbf{Simone Ciccolella} \\
    s.ciccolella@campus.unimib.it \\
    \textit{Mat. 762234}
  \end{minipage}%
  \begin{minipage}{.5\textwidth}
    \textbf{Daniele Bellani} \\
    d.bellani1@campus.unimib.it \\
    \textit{Mat. 780675}
  \end{minipage}
}
\end{minipage}
\vskip 1em \par
}
\makeatother
%
\begin{document}
%%

\title{ABSONES: Agent Based SOcial NEtwork Simulator}

\maketitle
%
\begin{abstract}
In questo progetto proponiamo un modello per la simulazione di social network, in particolare la piattaforma di microblogging Twitter. Il nostro approccio prevede la rappresentazione del sistema come un grafo diretto e l'identificazione degli utenti come agenti di tale sistema complesso, che si attivano e compiono azioni sulla base di probabilità ed estrazioni di tipo Monte Carlo. Gli obiettivi che ci siamo posti sono stati la prevenzione del \textit{completamento} della rete e il mantenimento del principio di \textit{invarianza di scala}, così da simulare, in modo quanto più fedele, il comportamento della vera rete sociale. Con il modello così costruito, è possibile studiare fenomeni sociologici usando scenari simulati nel modo più attinente possibile alla realtà. In particolare, abbiamo provato a simulare il ``salto" di popolarità di un individuo poco conosciuto in seguito all'interazione con un utente più ``in vista", traendo ispirazione da quanto avvenuto nel 2012 tra l'On. Maurizio Gasparri e l'utente Daniele Termite. 
\end{abstract}

%\title{ABSONES: Agent Based SOcial NEtwork Simulator}
%
%\author[1]{Simone Ciccolella}
%\author[2]{Daniele Bellani}
%\affil[1]{s.ciccolella@campus.unimib.it}
%\affil[2]{d.bellani1@campus.unimib.it}
%\setcounter{Maxaffil}{0}

%\date{\today}

%\begin{document}
%\maketitle
%
%\begin{abstract}
%In questo progetto proponiamo un modello per la simulazione di social network, in particolare la piattaforma di microblogging Twitter. Il nostro approccio prevede la rappresentazione del sistema come un grafo diretto e l'identificazione degli utenti come agenti di tale sistema complesso, che si attivano e compiono azioni sulla base di probabilità ed estrazioni di tipo Monte Carlo. Gli obiettivi che ci siamo posti sono stati la prevenzione del \textit{completamento} della rete e il mantenimento del principio di \textit{invarianza di scala}, in modo 
%\end{abstract}

\newpage
\tableofcontents
\newpage
\section{Introduzione}
\label{sec:introduction}

I social network sono uno strumento che è diventato parte integrante della vita di tutti. Nel giugno 2017 la piattaforma Facebook ha passato la soglia dei due miliardi di utenti attivi su base mensile \cite{facebook}; nel frattempo, Twitter si è imposto come mezzo di comunicazione principale tra i cosiddetti ``influencer", con un grosso impatto sull'opinione pubblica (per esempio, giocando un ruolo fondamentale nell'elezione di Donald Trump \cite{twitter}). L'aspetto principale di questa forma di comunicazione è la produzione, da parte degli utenti, di un'enorme mole di dati: le acquisizioni di YouTube da parte di Google e di LinkedIn da parte di Microsoft testimoniano l'interesse delle grandi aziende per queste sorgenti di dati. L'analisi di questi ultimi aiuta a determinare strategie di mercato, a personalizzare raccomandazioni di prodotti, studiare e prevedere il sentimento su un evento o un prodotto, oppure condurre studi di stampo sociologico. 
\\
\\
Sorgono però alcune difficoltà. In primo luogo, raramente questi dati vengono rilasciati in formato aperto (Open Data). Inoltre, le loro enormi proporzioni ne rendono difficile l'analisi e la gestione. Da qui la necessità di svolgere delle simulazioni \textit{in-silico} con modelli sviluppati in modo da essere il più possibile veritieri. L'attendibilità di queste simulazioni dipende dal modello e dalle inevitabili assunzioni che sono state fatte durante la costruzione.
\\
\\
Negli ultimi anni sono stati proposti alcuni modelli per la simulazione di social network, con risultati alterni e non definitivi, dovuti alla complessità del tema. Da questi abbiamo tratto ispirazione per il nostro modello. Nella prima sezione esamineremo alcuni dei modelli proposti, dopoichè passeremo all'illustrazione del nostro modello e della visione agent-based dello stesso. Nelle ultime due sezioni esporremo i risultati delle simulazioni e ne commenteremo le implicazioni.
\section{Theory 2-3 pages}
\label{sec:theory}

\subsection{Two-dimensional Electron Gas}
Here, explain the concept of a 2-DEG in GaAs/AlGaAs. What is a 2-DEG and why does it arise?

\subsection{Hall Effect}
Explain the classical Hall effect in your own words. What do I measure at $B=0$? And what happens if $B>0$? Which effect gives rise to the voltage drop in the vertical direction?

\subsection{Quantum Hall Effect}
Explain the IQHE in your own words. What does the density of states look like in a 2-DEG when $B=0$? What are Landau levels and how do they arise? What are edge states? What does the electron transport look like when you change the magnetic field? What do you expect to measure?

\section{Experiment 1-2 pages}
\subsection{Fabrication}
Explain a step-by-step recipe for fabrication here. How long did you etch and why? What is an Ohmic contact?
\subsection{Experimental set-up}
Explain the experimental set-up here. Use a schematic picture (make it yourself in photoshop, paint, ...) to show how the components are connected. Briefly explain how a lock-in amplifier works.

\section{Results and interpretation 2-3 pages}
Show a graph of the longitudinal resistivity ($\rho_{xx}$) and Hall resistivity ($\rho_{xy}$) versus magnetic field, extracted from the raw data shown in figure \ref{fig:data}. You will have the link to the data in your absalon messages, if not e-mail Guen (guen@nbi.dk). Explain how you calculated these values, and refer to the theory.

%\begin{figure}
%\centering
%\includegraphics[width=1\textwidth]{raw_data.png}
%\caption{\label{fig:data}Raw (unprocessed) data. Replace this figure with the one you've made, that shows the resistivity.}
%\end{figure}

\subsection{Classical regime}
Calculate the sheet electron density $n_{s}$ and electron mobility $\mu$ from the data in the low-field regime, and refer to the theory in section \ref{sec:theory}. Explain how you retrieved the values from the data (did you use a linear fit?).
Round values off to 1 or 2 significant digits: 8.1643 ~= 8.2. Also, 5e-6 is easier to read than 0.000005.

!OBS: This part is optional (only if you have time left).
Calculate the uncertainty as follows: \newline $u(f(x, y, z)) = \sqrt{(\frac{\delta f}{\delta{x}} u(x))^{2} + (\frac{\delta f}{\delta{y}} u(y))^{2} + (\frac{\delta f}{\delta{z}} u(z))^{2}}$, where $f$ is the calculated value ($n_{s}$ or $\mu$), $x, y, z$ are the variables taken from the measurement and $u(x)$ is the uncertainty in x (and so on).

\subsection{Quantum regime}
Calculate $n_{s}$ for the high-field regime.
Show a graph of the longitudinal conductivity ($\rho_{xx}$) and Hall conductivity($\rho_{xy}$) \textbf{in units of the resistance quantum} ($\frac{h}{e^{2}}$), depicting the integer filling factors for each plateau.
Show a graph of the plateau number versus its corresponding value of $1/B$. From this you can determine the slope, which you use to calculate the electron density.
Again, calculate the uncertainty for your obtained values.

\section{Discussion 1/2-1 page}
Discuss your results. Compare the two values of $n_{s}$ that you've found in the previous section. Compare your results with literature and comment on the difference. If you didn't know the value of the resistance quantum, would you be able to deduce it from your measurements? If yes/no, why?

\newpage
\section{Some LaTeX tips}
\label{sec:latex}
\subsection{How to Include Figures}

First you have to upload the image file (JPEG, PNG or PDF) from your computer to writeLaTeX using the upload link the project menu. Then use the includegraphics command to include it in your document. Use the figure environment and the caption command to add a number and a caption to your figure. See the code for Figure \ref{fig:frog} in this section for an example.

%\begin{figure}
%\centering
%\includegraphics[width=0.3\textwidth]{frog.jpg}
%\caption{\label{fig:frog}This frog was uploaded to writeLaTeX via the project menu.}
%\end{figure}

\subsection{How to Make Tables}

Use the table and tabular commands for basic tables --- see Table~\ref{tab:widgets}, for example.

\begin{table}
\centering
\begin{tabular}{l|r}
Item & Quantity \\\hline
Widgets & 42 \\
Gadgets & 13
\end{tabular}
\caption{\label{tab:widgets}An example table.}
\end{table}

\subsection{How to Write Mathematics}

\LaTeX{} is great at typesetting mathematics. Let $X_1, X_2, \ldots, X_n$ be a sequence of independent and identically distributed random variables with $\text{E}[X_i] = \mu$ and $\text{Var}[X_i] = \sigma^2 < \infty$, and let

\begin{equation}
S_n = \frac{X_1 + X_2 + \cdots + X_n}{n}
      = \frac{1}{n}\sum_{i}^{n} X_i
\label{eq:sn}
\end{equation}

denote their mean. Then as $n$ approaches infinity, the random variables $\sqrt{n}(S_n - \mu)$ converge in distribution to a normal $\mathcal{N}(0, \sigma^2)$.

The equation \ref{eq:sn} is very nice.

\subsection{How to Make Sections and Subsections}

Use section and subsection commands to organize your document. \LaTeX{} handles all the formatting and numbering automatically. Use ref and label commands for cross-references.

\subsection{How to Make Lists}

You can make lists with automatic numbering \dots

\begin{enumerate}
\item Like this,
\item and like this.
\end{enumerate}
\dots or bullet points \dots
\begin{itemize}
\item Like this,
\item and like this.
\end{itemize}
\dots or with words and descriptions \dots
\begin{description}
\item[Word] Definition
\item[Concept] Explanation
\item[Idea] Text
\end{description}

We hope you find write\LaTeX\ useful, and please let us know if you have any feedback using the help menu above.

\begin{thebibliography}{9}
\bibitem{facebook}
  Welch, Chris (June 27, 2017). "Facebook crosses 2 billion monthly users". \textit{The Verge. Vox Media.} Retrieved June 27, 2017.
  
\bibitem{twitter}
  Bulman, May (November 28, 2016). ``Donald Trump's 'celebrity-style' tweets helped him win US presidential election, says data scientist".
\textit{The Independent} Retrieved November 28, 2016.

\end{thebibliography}
\end{document}